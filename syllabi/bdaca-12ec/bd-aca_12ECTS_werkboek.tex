% !TEX encoding = UTF-8 Unicode

\documentclass[a4paper,10pt,twocolumn]{report}
\usepackage[natbibapa,nosectionbib,tocbib,numberedbib]{apacite}

% niet elk chapter op nieuwe pagina
%\let\cleardoublepage\relax \chapter{bar}

\AtBeginDocument{\renewcommand{\bibname}{Literature}}


\usepackage[utf8x]{inputenc}
\usepackage{graphicx}
\usepackage{enumerate}
\usepackage{url}

% corona boxes
\usepackage[dvipsnames]{color}
\usepackage{framed}
\definecolor{shadecolor}{named}{Melon}
%\let\corona\shaded
%\let\endcorona\endshaded
% zo kan het met sterretje:
\newenvironment{corona}{%
	\begin{shaded*}%
	}{%
	\end{shaded*}%
}



\usepackage[colorinlistoftodos]{todonotes}

\usepackage{pifont}

\usepackage{lmodern}
\usepackage{listings}
\lstset{
basicstyle=\scriptsize\ttfamily,
columns=flexible,
breaklines=true,
numbers=left,
%stepsize=1,
numberstyle=\tiny,
backgroundcolor=\color[rgb]{0.85,0.90,1}
}


\let\oldquote\quote
\let\endoldquote\endquote
\renewenvironment{quote}{\footnotesize\oldquote}{\endoldquote}



\title{Big Data and Automated Content Analysis\\ Part I and II (12 ECTS)\\~\\Course Manual}
%\author{dr. Damian Trilling \& dr.Anne Kroon \\~\\Graduate School of Communication\\University of Amsterdam\\~\\d.c.trilling@uva.nl | a.c.kroon@uva.nl \\www.damiantrilling.net\\@damian0604 | @annekroon \\~\\Office: REC-C, 8\textsuperscript{th} floor}
\author{dr. Damian Trilling\\~\\Graduate School of Communication\\University of Amsterdam\\~\\d.c.trilling@uva.nl \\www.damiantrilling.net \\@damian0604 \\~\\Office: REC-C, 8\textsuperscript{th} floor}


\date{Academic Year 2021/22\\Semester 2, block 1 and 2}



%Options: Sonny, Lenny, Glenn, Conny, Rejne, Bjarne, Bjornstrup
%\usepackage[Bjornstrup]{fncychap}


\begin{document}
\maketitle

%\tableofcontents


\chapter{About this course}

This course manual contains general information, guidelines, rules and schedules for the Research Master course Big Data \& Automated Content Analysis Part I and II (12 ECTS). Please make sure you read it carefully, as it  contains information regarding assignments, deadlines and grading.

\section{Course description}


``Big data'' is a relatively new phenomenon, and refers to data that are more voluminous, but often also more unstructured and dynamic, than traditionally the case. In Communication Science and the Social Sciences more broadly, this in particular concerns research that draws on Internet-based data sources such as social media, large digital archives, and public comments to news and products This emerging field of studies is also called \emph{Computational Social Science} \citep{Lazer2009} or even \emph{Comutational Communication Science} \citep{Shah2015}.

%One of the big challenges is being able to derive information from these data that can be handled meaningfully and economically at the same time.

The course will provide insights in the concepts, challenges and opportunities associated with data so large that traditional research methods (like manual coding) cannot be applied any more and traditional inferential statistics start to loose their meaning. Participants are introduced to strategies and techniques for capturing and analyzing digital data in communication contexts. We will focus on (a) data harvesting, storage, and preprocessing and (b) computer-aided content analysis, including natural language processing (NLP) and computational social science approaches. In particular, we will use advanced machine learning approaches and models like word embeddings.

To participate in this course, students are expected to be interested in learning how to write own programs where off-the-shelf software is not available. Some basic understanding of programming languages is helpful, but not necessary to enter the course. Students without such knowledge are encouraged to follow a (free) online course (such as the one at \url{https://www.codecademy.com/learn/python}) to prepare.


\section{Goals}
Upon completion of this course, the following goals are reached:
\begin{enumerate}[A]
\item Students can explain the research designs and methods employed in existing research articles on Big Data and automated content analysis.
\item Students can on their own and in own words critically discuss the pros and cons of research designs and methods employed in existing research articles on Big Data and automated content analysis; they can, based on this, give a critical evaluation of the methods and, where relevant, give advice to improve the study in question.
\item Students can identify research methods from computer science and computational linguistics which can be used for research in the domain of communication science; they can explain the principles of these methods and describe the value of these methods for communication science research.
\item Students can on their own formulate a research question and hypotheses for own empirical research in the domain of Big Data.
\item Students can on their own chose, execute and report on advanced research methods in the domain of Big Data and automatic content analysis.
\item Students know how to collect data with scrapers, crawlers and APIs; they know how to analyze these data and to this end, they have basic knowledge of the programming language Python and know how to use Python-modules for communication science research.
\item Students can critically discuss  strong and weak points of their own research and suggest improvements.
\item Students participate actively: reading the literature carefully and on time, completing assignments carefully and on time, active participation in discussions, and giving feedback on the work of fellow students give evidence of this.
\end{enumerate}

\section{Help with practical matters}
While making your first steps with programming in Python, you will probably have a lot of questions.
Nevertheless, \url{htttps://google.com} and \url{https://stackoverflow.com} should be your first points of contact. After all, that's how we solve our problems as well\ldots

It is also very much encouraged to try to solve problems together with your classmates. A succesful technique can be \emph{pair programming}, where two colleagues sit behind one screen (for an explanation, see \url{https://en.wikipedia.org/wiki/Pair_programming}).

If you really cannot find a solution on your own, you can pose such questions during the Friday lab sessions. However, keep the remark in Section~\ref{sec:yourcomputer} in mind!


\chapter{Rules, assignments, and grading}
The final grade of this course will be composed of the grade of two mid-term take home exam ($2 \times 20$\%) and one individual project (60\%).

\section{Mid-term take-home exams ($2 \times 20$\%)}
In two mid-term take-home exam, students will show their understanding of the literature and prove they have gained new insights during the lectures and lab sessions. They will be asked to critically assess various approaches to Big Data analysis and make own suggestions for research.

\section{Final individual project ($60\%$)}
The final individual project typically consists of the following elements:
\begin{itemize}
\item a carefully collected and relevant dataset of non-trivial size;
\item a set of scripts for collecting, preprocessing, and analyzing the data. The scripts should be well-documented and tailored to the specific needs of the own project;
\item output files;
\item a report with
\begin{itemize}
	\item introduction including references to relevant (course) literature, an overarching research question plus subquestions and/or hypotheses (1--2 pages);
	\item an overview of the analytic strategy, referring to relevant methods learned in this course;
	\item an answer to the RQ;
	\item a well-substantiated conclusion and directions for future research.
\end{itemize}
\end{itemize}

The project should cover the majority of techniques from the course, unless the approaches contradict each other (e.g., it makes sense to focus on either supervised or unsupervised approaches; data may be either text or image; one may use either webscraping or an API).
The project has to
\begin{itemize}
\item include techniques from \emph{both} Part I and Part II of the course;
\item aim to use the most appropriate technique in case of alternatives. In particular, this means that introductory examples that are discussed for educational purposes only may not be sufficient, if better alternatives ave been discussed in the course as well. 
\end{itemize}
If in doubt, discuss with me in advance!




\section{Grading and 2\textsuperscript{nd} try}
Students have to get a pass (5.5 or higher) for both mid-term take-home exams and the individual project. If the grade of one of the mid-term exams is lower, a re-sit will be organized, typically ten days after the grade is communicated to the student.
If the grade of the final project is lower, an improved version can be handed in within ten days after the grade is communicated to the student. If the improved version still is graded lower than 5.5, the course cannot be completed. Improved versions of the final individual project cannot be graded higher than 6.0.

In any case of a second try, the grade of the second try replaces the grade of the first try, regardless of whether it is higher or lower than the first try.

\section{Presence and participation}
Attendance is compulsory. Missing more than three meetings – for whatever reason – means the course cannot be completed.

Next to attending the meetings, students are also required to prepare the assigned literature and to continue working on the programming tasks after the lab sessions. To successfully finish the course, attending the lab sessions is not enough, but has to go hand-in-hand with continuous self-study.

\section{Staying informed}
It is your responsibility to check the means of communications used for this course (i.e., your email account, but -- if applicable -- also e-learning platforms or any other tool that the lecturer decides to use) on a regular basis, which in most cases means daily.

\section{Plagiarism \& fraud}
Plagiarism is a serious academic violation. Cases in which students use material such as online sources or any other sources in their written work and present this material as their own original work without citation/referencing, and thus conduct plagiarism, will be reported to the Examencommissie of the Department of Communication without any further negotiation. If the committee comes to the conclusion that a student has indeed committed plagiarism the course cannot be completed.
General UvA regulations about fraud and plagiarism apply.

\subsection{Attributing code}
Plagiarism rules apply to code in the same way as they apply to text. Just like how your writing often builds on other scholars (who you cite), your code often builds on what others have done or are inspired by solutions of others, for instance in communities like Stackoverflow. Analoguous to plagiarism regulations in normal writing, also in your code, it needs to be transparent \emph{what is from you and what is not}. We don't really have something like APA references for code, but you can simply use \texttt{\#comments} for that purpose, such as:
\texttt{\# The following function is copied from https://stackoverflow.com/XXXXX/XXXXX} for (almost) literal copy-pasting, or \texttt{\# The code in this cell is inspired by https://...; I modified Y and Z}. 

Make sure that there is no doubt about what is yours and what is from someone else!

\subsection{Working together}
The take-home exams as well as the final project are \emph{individual exams}. Collaboration with fellow students is thus considered academic fraud and will be reported and handled accordingly.

In some instances, some overlap between final projects may be unavoidable, for instance when relying on very similar data sources. If in doubt, always discuss with me in advance to avoid problems after you handed in! It is almost always possible to find a solution, if discussed in advance.

\emph{In contrast to these graded assignments, for all \textbf{ungraded} assignments and exercises, collaboration is explicitly encouraged. Experience with this course over the last decade has shown that those students who met regularly with classmates to go over the week's code examples and exercises together enjoyed the course most, and also performed best.}

\section{Deadlines and handing in}
Per assignment, the lecturer will specify whether it has to be handed in via Canvas or via Filesender. If no specific instructions are given, or if there is any issue with submitting an assignment through Canvas, use Filesender. Please send all assignments and papers as a PDF file (and do not use formats like .docx) to ensure that it can be read and is displayed the same way on any device. Multiple files should be compressed and handed in as one .zip file or .tar.gz file. Do not email assignments directly but send them via \url{https://filesender.surf.nl/} to my mail-address. This way, you can also transfer huge files.

Final papers and take-home exams that are not handed in on time, will be not be graded and receive the grade 1. This rule also applies for any other assignment that might be given. The deadline is only met when the all files are submitted.


\begin{corona}
\section{Your computer and operating system: your responsibility}	
\label{sec:yourcomputer}
Python is a language, not a program, and there are many ways of running Python code. 

It is really hard to help with stuff related to your computer, your operating system, your keyboard etc. Things may work slightly different on Windows compared to MacOS or Linux; it's your responsibility to take care of these things. For instance, a downloaded file may be put under \texttt{/home/damian/Downloads} on Linux, under \texttt{/Users/damian/Downloads} on MacOS, and under \texttt{c:\textbackslash \textbackslash Users \textbackslash \textbackslash Damian\textbackslash \textbackslash Downloads} on Windows -- or maybe something very different. You are expected to have a sufficiently deep understanding of your own computer to deal with these idiosyncrasies yourself.

Therefore you need to make sure that the following things apply. Otherwise, you need to fix it -- we won't have room for that during the course. So, do this \textbf{before the course starts}.

\begin{itemize}
\item You need to have a recent version of a Python interpreter installed, and you need to be able to run Jupyter Notebooks. Chapter 1 of \cite{cssbook} gives instructions (you can skip the part about R, though). You can either use Python natively or use Anaconda (see Chapter 1).
	\item You need to be able to install third-party packages (see Chapter 1). Test this by checking whether you can install the package \texttt{shifterator} and then load it with \texttt{import shifterator} without getting an error.
	\item You need to make sure that your keyboard produces straight quotes \texttt{"} rather than typographical ones when typing. Test that by making sure that you can run \texttt{print("hello world")} in a Jupyter notebook.
	\item Create a folder for this course at an easy to remember place, such as \texttt{/home/damian/bdaca} (Linux), \texttt{/Users/damian/bdaca} (MacOS), or \texttt{c://Users//damian//bdaca} (Windows). You could potentially also use \texttt{c://Users//damian//Desktop//bdaca} (Win) or \texttt{/Users/damian/Desktop/bdaca} (Mac) or similar. Make sure that you can locate files in that folder in Jupyter Notebook. And remember/write down it's name! Preferably, don't use spaces (can lead to confusion), and Windows users, replace \texttt{\textbackslash} by either a double \texttt{\textbackslash \textbackslash} or by forward slashes (/).
\end{itemize}
If -- even after trying -- you do not succeed in any of these steps, ask your classmates or make an appointment with Vladislav as soon as possible.


\end{corona}




\chapter{Schedule and Literature}


The following schedule gives an overview of the topics covered each week, the obligatory literature that has to be studied each week, and other tasks the students have to complete in preparation of the class.
In particular, the schedule shows which chapter of \cite{cssbook} will be dealt with. Note that some basic chapters that explain how to install the software we are going to use have to be read before the course starts.

Next to the obligatory literature, the following books provide the interested student with more and deeper information. They are intended for the advanced reader and might be useful for final individual projects, but are by no means required literature. Bear in mind, though, that you may encounter slightly outdated examples (e.g., Python 2, now-defunct APIs etc.).

\begin{itemize}
%\item \citealp{Russel2013}. Gives a lot of examples about how to analyze a variety of online data, including Facebook and Twitter, but going much beyond that.
%\item \citealp{Bird2009}. This is the official documentation of the NLTK package that we are using. A newer version of the book can be read for free at \url{http://nltk.org}
\item \citealp{McKinney2012}: A lot of examples for data analysis in Python. A PDF of the book can be downloaded for free on \url{http://it-ebooks.info/book/1041/}.
\item \citealp{VanderPlas2016}: A more recent book on numpy, pandas, scikit-learn and more. It can also be read online for free on \url{https://jakevdp.github.io/PythonDataScienceHandbook/}, and the contents are avaibale as Jupyter Notebooks as well \url{https://github.com/jakevdp/PythonDataScienceHandbook}.
\item The pandas cookbook by Julia Evans, a collection of notebooks on github: \url{https://github.com/jvns/pandas-cookbook}.
\item \citealp{Hovy2020}: A thin book on bottom-up text analysis in Python with both a bit more math background and ready-to-use Python code implementations.
\item \citealp{Salganik2017}: Not a book on Python, but on research methods in the digital age. Very readable, and a lots of inspiration and background about techniques covered in our course.
\end{itemize}


\section*{Before the course starts: Prepare your computer.}
\textsc{\ding{52} Chapter 1: Introduction}\\
Make sure that you have a working Python environment installed on your computer. You cannot start the course if you have not done so.

\section*{PART I: Basics of Python and ACA}

\section*{Week 1: What is Computational Social Science, and why Python?}
\subsection*{Wednesday, 9--2. Lecture.}
We discuss what Big Data and Computational (Social\textbar Communication) Science are. We talk about challenges and opportunities as well as the implications for the social sciences in general and communication science in particular. We also pay attention to the tools used in CSS, in particular to the use of Python.

Mandatory readings (in advance):  \cite{boyd2012}, \cite{Kitchin2014}, \cite{Hilbert2019}.

Additionally, the journal \textit{Commmunication Methods and Measures} had a special issue (volume 12, issue 2--3) about Computational Communication Science. Read at least the editorial \citep{VanAtteveldt2018a}, but preferably, also some of the articles (you can also do that later in the course).


\subsection*{Friday, 11--2. Lab session.}
\textsc{\ding{52} Chapter 2: Fun with data}\\

During the lab session, we will run our first code. We will showcase some possibilities, and leave the technical background for next week. Also, we will discuss different ways of running Python code. In particular, we will get used to working with classic Jupyter Notebooks and JupyterLab. We will also briefly look at other ways of running Python code. At the end of the session, you should have developed a workflow that works \emph{for you}.


%%%%%%%%%%%%%%%%%%%%%%%%%%%%%%%%%%%%%%%%%%%%%%%%%%%%%%%%%%%%%%%%%%%%%%%%%%%%%%%%%%%%%

\section*{Week 2: Programming concepts and writing code}

\subsection*{Wednesday, 16--2. Lecture.}
\textsc{\ding{52} Chapter 3: Programming concepts for data analysis}\\
You will get a very gentle introduction to computer programming. During the lecture, you are encouraged to follow the examples on your own laptop.


\subsection*{Friday, 18--2. Lab session.}
\textsc{\ding{52} Chapter 4: How to write code}\\
We will do our first real steps in Python and do some exercises to get the feeling.


%%%%%%%%%%%%%%%%%%%%%%%%%%%%%%%%%%%%%%%%%%%%%%%%%%%%%%%%%%%%%%%%%%%%%%%%%%%%%%%%%%%%%

\section*{Week 3: From files and APIs to lists, dictionaries, or dataframes}
We talk about file formats such as \texttt{csv} and \texttt{json}; about encodings; about reading these formats into basic Python structures such as dictionaries and lists as opposed to reading them into dataframes; and about retrieving such data from local files, as parts of packages, and via an API.


\subsection*{Wedneday, 23--2. Lecture.}
\textsc{\ding{52} Chapter 5: From file to dataframe and back}\\
A conceptual overview of different file formats and data sources, and some practical guidance on how to handle such data in basic Python and in Pandas.

\subsection*{Friday, 25--2. Lab session.}
\textsc{\ding{52} Chapter 12.1: Using web APIs: from open resources to Twitter}\\
We will write a script to collect and handle some JSON data.


%%%%%%%%%%%%%%%%%%%%%%%%%%%%%%%%%%%%%%%%%%%%%%%%%%%%%%%%%%%%%%%%%%%%%%%%%%%%%%%%%%%%%


\section*{Week 4: Data wrangling and exploratory data analysis}
Of course, you don't need Python to do statistics. Whether it's R, Stata, or SPSS -- you probably already have a tool that you are comfortable with. But you also do not want to switch to a different environment just for getting a correlation. And you definitly don't want to do advanced data wrangling in SPSS\ldots
This week, we will discuss different ways of organizing your data (e.g., long vs wide formats) as well as how to do conventional statistical tests and simple plots in Python.


\subsection*{Wednesday, 2--3. Lecture}
\textsc{\ding{52} Chapter 6: Data wrangling}\\
\textsc{\ding{52} Chapter 7.1. Simple exploratory data analysis}\\
\textsc{\ding{52} Chapter 7.2. Visualizing data}\\

We will learn how to get your data in the right shape and how to get a first understanding of your data, using exploratory analysis and visualization techniques. We will cover data wrangling with pandas: converting between wide and long formats (melting and pivoting), aggregating data, joining datasets, and so on.


\subsection*{Friday, 4--3. Lab session.}
We will apply the techniques discussed during the lectures to multiple datasets.


%%%%%%%%%%%%%%%%%%%%%%%%%%%%%%%%%%%%%%%%%%%%%%%%%%%%%%%%%%%%%%%%%%%%%%%%%%%%%%%%%%%%%


\section*{Week 5: Statistical Modeling and Machine Learning}
In this week, we will make the transition from classic statistical modeling as you know it from your previous courses to machine learning. We will discuss how both approaches are related  (or even identical) and where the differences are.


\subsection*{Wednesday, 9--3. Lecture.}

\textsc{\ding{52} Chapter 7.3. Clustering and Dimensionality Reduction}\\
\textsc{\ding{52} Chapter 8: Statistical Modeling and Supervised Machine Learning}\\
\textsc{\ding{54} (you can skip 8.4 Deep Learning for now)}\\

We will discuss what unsupervised and supervised machine learning are, what they can be used for, and how they can be evaluated.

\subsection*{Friday, 11--3. Lab session.}
Departuring from a brief encounter with statsmodels \citep{statsmodels}, a library for statistical modelling, you will learn how to work with scikit-learn \citep{scikit-learn}, one of the most well-known machine learning libraries.

\subsection*{Take-home exam}
In week 5, the first midterm take-home exam is distributed after the Friday meeting. The answer sheets and all files have to be handed in no later than the day before the next meeting, i.e. Tuesday evening (15--5, 23.59).

%%%%%%%%%%%%%%%%%%%%%%%%%%%%%%%%%%%%%%%%%%%%%%%%%%%%%%%%%%%%%%%%%%%%%%%%%%%%%%%%%%%%%




\section*{Week 6: Processing textual data}
In this week, we will dive into how to deal with textual data. How is text represented, how can we process it, and how can we extract useful information from it?
Unfortunately, text as written by humans usually is pretty messy.  We will therefore dive  into ways to represent text in a clean(er) way. We will introduce the Bag-of-Words (BOW) representation and show multiple ways of transforming text into matrices.


\subsection*{Wedneday, 16--3. Lecture.}
\textsc{\ding{52} Chapter 9: Processing text}\\
\textsc{\ding{52} Chapter 10: Text as data}\\
\textsc{\ding{52} Chapter 11: Automatic analysis of text}\\
\textsc{\ding{54} (you can skip 11.4 and 11.5 for now)}\\
  
This lecture will introduce you to techniques and concepts like lemmatization, stopword removal, n-grams, word counts and word co-occurrances, and regular expressions. We then proceed to introducing BOW representations of text.

Additional recommended background reading on stopwords: \cite{Nothman2018}.


\subsection*{Friday, 18--3. Lab session.}
You will combine the techiques discussed on Wednesday and write a first automated content analysis script.





%%%%%%%%%%%%%%%%%%%%%%%%%%%%%%%%%%%%%%%%%%%%%%%%%%%%%%%%%%%%%%%%%%%%%%%%%%%%%%%%%%%%%



\section*{Week 7: Web scraping}
Reserve some time for exercising in this week. Web scraping can be really hard, because there are so many specifics of specific websites to consider. After all, every website is different, and we need to customize scrapers for every site! At the same time, it is one of the most useful techniques to know, and the majority of students in previous cohorts used web scraping as a part of their final project.


\subsection*{Wednesday, 23--3. Lecture.}
\textsc{\ding{52} Chapter 12: Scraping online data}\\
We will explore techniques to download data from web pages and to extract meaningful information like the text (or a photo, or a headline, or the author) from an article on \url{http://nu.nl}, a review (or a price, or a link) from \url{http://kieskeurig.nl}, or similar.

\subsection*{Friday, 25--3. Lab session.}
We will exercise with web scraping and parsing.




%%%%%%%%%%%%%%%%%%%%%%%%%%%%%%%%%%%%%%%%%%%%%%%%%%%%%%%%%%%%%%%%%%%%%%%%%%%%%%%%%%%%%


\section*{Break between block 1 and 2}


%%%%%%%%%%%%%%%%%%%%%%%%%%%%%%%%%%%%%%%%%%%%%%%%%%%%%%%%%%%%%%%%%%%%%%%%%%%%%%%%%%%%%


\section*{PART II: Advanced analyses}


\section*{Week 8: Supervised Approaches to Text Analysis}
\textsc{\ding{52} Chapter 11: Automatic analysis of text}\\
\textsc{\ding{54} (you can skip 11.5. Unsupervised text analysis: topic modeling for now)}\\


\subsection*{Wednesday, 6--4. Lecture.}

We discuss why and when to choose supervised machine learning approaches as opposed to dictionary- or rule-based approaches, and explore how BOW representations can be used as an input for supervised machine learning.

Mandatory reading: \cite{Boumans2016}.

\subsection*{Friday, 8--4. Lab session.}
Exercises with scikit-learn.


%%%%%%%%%%%%%%%%%%%%%%%%%%%%%%%%%%%%%%%%%%%%%%%%%%%%%%%%%%%%%%%%%%%%%%%%%%%%%%%%%%%%%

\section*{Week 9: Supervised Approaches to Text Analaysis II}

\subsection*{Wednesday, 13--4.}

We will continue with the topic in week 8, with special attention on how to find the best model using techniques such as crossvalidation and gridsearch.


\subsection*{Friday, 15--4. No meeting (Good Friday).}


%%%%%%%%%%%%%%%%%%%%%%%%%%%%%%%%%%%%%%%%%%%%%%%%%%%%%%%%%%%%%%%%%%%%%%%%%%%%%%%%%%%%%


\section*{Week 10: Unsupervised Machine Learning for Text}
\textsc{\ding{52} Chapter 11.5. Unsupervised text analysis: Topic modeling}\\

\subsection*{Wednesday, 20-4. Lecture.}
Mandatory readings (in advance): \cite{Maier2018a} and \cite{Tsur2015}.

We will discuss Latent Dirichlet Allication (LDA) topic models. We will contrast them with other unsupervised machine learning approaches such as principal component analysis, k-means clustering, and hiearchical clustering.


\subsection*{Friday, 22-4. Lab session.}
You will apply the techniques discussed on Wednesday using gensim \citep{Rehurek2010}.



%%%%%%%%%%%%%%%%%%%%%%%%%%%%%%%%%%%%%%%%%%%%%%%%%%%%%%%%%%%%%%%%%%%%%%%%%%%%%%%%%%%%%



\section*{Week 11: Beyond Bag-of-Words}

\subsection*{Wednesday, 27--4. No lecture: Koningsdag}

\subsection*{Friday, 29--4. Lecture (!)}
\textsc{\ding{52} Chapter 10.3.3. Word Embeddings}\\
\textsc{\ding{52} Chapter 8.3.5. Neural Networks}\\
\textsc{\ding{52} Chapter 8.4. Deep Learning}\\

In this week, we will talk about a problem of standard forms of ACA: they treat words as independent from each other, and as either present or absent. For instance, if ``teacher'' is a feature in a specific model, and a text mentions ``instructor'', then this is not captured -- even though it probably should matter, at least to some extend. Word embeddings are a technique to overcome this problem. But also, they can reveal hidden biases in the texts they are trained on.

We will ask whether we still should use BOW approaches at all? Are LDA topic models still a good choice, or do we have to move towards using fancy language models such as BERT? And is that also true for classification tasks?


%Contextualized Topic Models (CTM), topic models with embeddings and BERT: https://github.com/MilaNLProc/contextualized-topic-models

%https://towardsdatascience.com/interactive-topic-modeling-with-bertopic-1ea55e7d73d8

%https://github.com/MaartenGr/BERTopic


Mandatory readings (in advance): \cite{Kusner2015} and \cite{Garg2017}.


You will also be provided with examples for how to apply a word2vec model and get a short introduction to keras.




\subsection*{Take-home exam}
In week 11, the second midterm take-home exam is distributed after the Friday meeting. The answer sheets and all files have to be handed in no later than the day before the next meeting, i.e. Tuesday evening (10--5, 23.59).

% THIS MEANS THEY HAVE MORE THAN A WEEK THIS TIME -- BUT OK, WHY NOT?


%%%%%%%%%%%%%%%%%%%%%%%%%%%%%%%%%%%%%%%%%%%%%%%%%%%%%%%%%%%%%%%%%%%%%%%%%%%%%%%%%%%%%


\section*{Week 12: No Teaching (UvA Teaching Free Week)}

%\noindent \textsc{\ding{52} Chapter 13 Network data}\\



%%%%%%%%%%%%%%%%%%%%%%%%%%%%%%%%%%%%%%%%%%%%%%%%%%%%%%%%%%%%%%%%%%%%%%%%%%%%%%%%%%%%%



\section*{Week 13: Multimedia data}

\subsection*{Wednesday, 11-5. Lecture}
\textsc{\ding{52} Chapter 14 Multimedia data}\\
We will look beyond text and discuss approaches to the computational analysis of multimedia data.


\subsection*{Friday, 13-5. Lab Session}
Opportunity to exercise with APIs and libraries presented during the lecture.




\section*{Week 14: Wrapping up}

\subsection*{Wendesday, 18--5. Guest Lecture by Luna De Bruyne}
In this guest lecture, Luna De Bruyne, a computational linguist from Ghent University, will share with us her work on modern emotion detection and sentiment analysis using the latest techniques and insights from computational linguistics.


In the second part of the lecture, we will look back and look forward. We try to bring all parts we learned together and discuss how to combine all parts into a good CCS-project. We will also focus on best practices and discuss alternatives to approaches we covered in this course. 



\subsection*{Friday, 20--5. Open Lab}
Open meeting with the possibility to ask last (!) questions regarding the final project.


\subsection*{Final project}
Deadline for handing in: Friday, 3--6, 23.59.


\bibliographystyle{apacite}
\bibliography{../../resources/literature}




\end{document}
