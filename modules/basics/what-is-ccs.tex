%What is computational \lbrack social\textbar communication\rbrack  ~science?



\subsection{Defining CCS}



\begin{frame}{A very young field}
	\begin{block}{\textcite{Lazer2009}}
		``The capacity to collect and analyze massive amounts of data has transformed such fields as biology and physics. But the emergence of a data-driven `computational social science' has been much slower.''
	\end{block}
\end{frame}




\begin{frame}{Epistemologies and paradigm shifts}
	\begin{block}{\textcite{Kitchin2014}}<1->
		\begin{itemize}
			\item<2-> (Reborn) empiricism: purely inductive, correlation is enough
			\item<3-> Data-driven science: knowledge discovery guided by theory
			\item<4-> Computational social science and digital humanities: employ Big Data research within existing epistemologies
			\begin{itemize}
				\item DH: descriptive statistics, visualizations
				\item CSS: prediction and simulation
			\end{itemize}
		\end{itemize}
	\end{block}
\end{frame}




\begin{frame}{CCS as a subset of CSS}
	\begin{block}{\textcite{Hilbert2019}}
		``\ldots our definition of computational communication science as an application of computational science to questions of human and social communication. As such, it is a natural subfield of computational social science'' (followed by references to CSS definitions)
	\end{block}
\end{frame}



\begin{frame}{Data, analysis, theory}
	\begin{block}{\textcite{VanAtteveldt2018a}}
		``\ldots computational communication science studies generally involve: (1) large and complex data sets; (2) consisting of digital traces and other ``naturally occurring'' data; (3) requiring algorithmic solutions to analyze; and (4) allowing the study of human communication by applying and testing communication theory.''
	\end{block}	
	
\end{frame}





\subsection{And Big Data?}

\begin{frame}[standout]
It was a buzzword when we first designed this course in 2013 (very much like ``AI'' and ``algorithm'' are today).

It's hard to define, but it can help us thinking about the characteristics of the data we deal with.
\end{frame}


\begin{frame}{The ``pragmatic'' definition }
	\begin{block}{}
		Everything that needs so much computational power and/or storage that you cannot do it on a regular computer.
	\end{block}
\end{frame}



\begin{frame}{The ``commercial'' definition }
	\begin{block}{\textcite{gartner}}
		``Big data is high-volume, high-velocity and/or high-variety information assets that demand cost-effective, innovative forms of information processing that enable enhanced insight, decision making, and process automation.''
	\end{block}
\end{frame}



\begin{frame}{The ``critical'' definition }
	\begin{block}{\textcite{boyd2012}}
		``
		\begin{enumerate}
			\item Technology: maximizing computation power and algorithmic accuracy to gather, analyze, link, and compare large data sets.
			\item Analysis: drawing on large data sets to identify patterns in order to make economic, social, technical, and legal claims.
			\item Mythology: the widespread belief that large data sets offer a higher form of intelligence and knowledge that can generate insights that were previously impossible, with the aura of truth, objectivity, and accuracy.
		\end{enumerate}
		''
	\end{block}
\end{frame}




\question{Do you think we are doing Big Data analysis?}



\question{
	\begin{enumerate}
		\item 	What do you think? What is the essence of Big Data/CSS/CCS?
		\item How will what we do here relate to theories and methods from other courses?
	\end{enumerate}
}





\subsection{The role of software in CSS}

\begin{frame}{CSS means also a shift in our relationship to software}
	In the (traditional) social sciences, we tend to think of software as\ldots
\begin{itemize}
	\item a monolithic block: a large standalone programme like Word or SPSS;
	\item something that either fulfills your needs or doesn't: it's often impossible, but at least uncommon and difficult, to adapt or change it;
	\item something with a graphical user interface;
	\item (often) something you pay for.
\end{itemize}
\end{frame}



\begin{frame}{CSS means also a shift in our relationship to software}
	In CSS, that's (typically) different:
	\begin{itemize}
		\item We mix and combine different ``modules'' or ``libraries'' ($\approx$ building blocks for programs)
		\item We build on modules made by others so that we don't reinvent the wheel, but ultimately build our own programs.
		\item We use a programming language to do all of this.
	\end{itemize}
\end{frame}






\begin{frame}{Why program your own tool?}
	\begin{block}{\textcite{Vis2013}}
		``Moreover, the tools we use can limit the range of questions that might be imagined, simply because they do not fit the affordances of the tool. Not many researchers themselves have the ability or access to other researchers who can build the required tools in line with any preferred enquiry. This then introduces serious limitations in terms of the scope of research that can be done.''	
	\end{block}
	
\end{frame}


\begin{frame}{Some considerations regarding the use of software in science}
	Assuming that science should be \emph{transparent} and \emph{reproducible by anyone}\onslide<2->{, we should}
	\begin{block}{use tools that are}<2->
		\begin{itemize}
			\item platform-independent 
			\item free (as in beer and as in speech, gratis and libre)
			\item which implies: open source
		\end{itemize}
	\end{block}
	\onslide<3>{This ensures it can our research (a) can be reproduced by anyone, and that there is (b) no black box that no one can look inside. $\Rightarrow$ ongoing open-science debate! \parencite{VanAtteveldt2019}}
\end{frame}

\begin{frame}{Why program your own tool?}
	\begin{block}{\textcite{Vis2013}}
		``{[}\ldots{]} these {[}commercial{]} tools are often unsuitable for academic purposes because of their cost, along with the problematic `black box' nature of many of these tools.''
	\end{block}
	
	\begin{block}{\textcite{Mahrt2013}}
		``{[}\ldots{]} we should resist the temptation to let the opportunities and constraints of an application or platform determine the research question {[}\ldots{]}''
	\end{block}
\end{frame}


\question{Do you think one \textit{needs} a programming language to do CSS? Why?}

\begin{frame}{So I need a programming language to do CSS?}

\begin{itemize}
	\item In principle, it does not matter \emph{which} programming language you use.
	\item \textbf{Python} and \textbf{R} are by far the most popular languages \emph{for CSS}
	\item But that may change: Nowadays, most scientific articles are written in English -- in some fields, that used to be Latin, German, Russian
\end{itemize}

\textbf{If you learn one programming language, it is relatively easy to learn another one.}
(Think of learning Spanish after you learned French)

\end{frame}
